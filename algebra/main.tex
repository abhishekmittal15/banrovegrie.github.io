\documentclass{article}
\usepackage[utf8]{inputenc}
\usepackage{amsmath} % for 'bmatrix' environment
\usepackage{graphicx}
\usepackage{subfiles}
\usepackage{amssymb}
\usepackage{fullpage}
\graphicspath{ {./inv/} }

\usepackage[english]{babel}
\usepackage[protrusion=true,expansion=true]{microtype}	
\usepackage{amsmath,amsfonts,amsthm,amssymb}
\usepackage{graphicx}
\usepackage{booktabs}


\title{MA3.101: Linear Algebra Final Assignment}
\date{4th May, 2020}
\begin{document}

\maketitle{This Document contains Solutions to the Problems posed to \textbf{Group A4} as a part of LA Final Assignment}

\section*{Authors}

\begin{itemize}
    \item \textbf{Tushar Choudhary}, 2019111019\\
    \item \textbf{Tathagato Roy}, 2019111020\\
    \item \textbf{Megha Bose}, 2019111021\\
    \item \textbf{Prateek Kumar Patel}, 2019111022\\
    \item \textbf{Alapan Chaudhuri}, 2019111023\\
    \item \textbf{Zeeshan Ahmed}, 2019111024\\
\end{itemize}

\tableofcontents
\begin{enumerate}
    \item Solution to Problem 1\ \ \        ---------------------------------------------------------------------------------------------\ \ \ (2)
    \item Solution to Problem 2\ \ \        ---------------------------------------------------------------------------------------------\ \ \ (4)
\end{enumerate}

\newpage
\section*{Problem 1}
Prove that for $m \times n$ matrices $A$ and $B$ , $rank(A+B) \leq rank(A) +rank(B)$. (Hint (use if necessary): Think of two linear transformations induced by the matrices, which have ranks $rank(A+B)$ and $rank(A)+rank(B)$ respectively. Then compare their col space (or row space))
\subsection*{Solution}
Consider the following two matrices A and B of dimension $m \times n$

$$
    A =
    \left[
    \begin{matrix}
        a_1 \\
        a_2 \\
        .\\
        .\\
        .\\
        a_m\\
    \end{matrix}
    \right]\ \ \ \ \ 
    B =
    \left[
    \begin{matrix}
        b_1 \\
        b_2 \\
        .\\
        .\\
        .\\
        b_m\\
    \end{matrix}
    \right]\\
$$

where, $\forall\ i \in \{1,2,3,...,m\},$ \textbf{$a_i$} and \textbf{$b_i$} are matrices of dimension $1 \times n$.\\

Then, the rank of the matrix A is the dimension of the column space of A and we have 
\[
rank(A) = dim(span(\{a_1, a_2, ..., a_m\})) \tag{1}
\]

\[
rank(B) = dim(span(\{b_1, b_2, ..., b_m\})) \tag{2}
\]\\

Similarly, since
$$
    A + B =
    \left[
    \begin{matrix}
        a_1+b_1 \\
        a_2+b_2 \\
        .\\
        .\\
        .\\
        a_m+b_m\\
    \end{matrix}
    \right]\\
$$

we also have 

\[
rank(A+B) = dim(span(\{(a_1+b_1), (a_2+b_2), ..., (a_m+b_m)\})) \tag{3}
\]\\

Let the vector $\Vec{c} \in span(\{(a_1+b_1), (a_2+b_2), ..., (a_m+b_m)\})$ then

\begin{equation}
\Vec{c} = \sum\limits_{i=1}^{m} r_i(a_i+b_i) \tag{4}
\implies \Vec{c} = \sum\limits_{i=1}^{m} r_ia_i + \sum\limits_{i=0}^{m} r_ib_i
\end{equation}\\
\[
where\ \forall\ i \in \{1,2,3,...,m\},\ r_i \in \mathbb{C}
\]
\newpage

Thus, $\Vec{c} \in span(\{a_1, a_2, ..., a_m\}) + span(\{b_1, b_2, ..., b_m\})$ and hence

\[
span(\{a_1+b_1, a_2+b_2, ..., a_m+b_m\})
\subseteq span(\{a_1, a_2, ..., a_m\}) + span(\{b_1, b_2, ..., b_m\})
\]

\begin{center}
\begin{multline}
\implies  dim(span(\{(a_1+b_1), (a_2+b_2), ..., (a_m+b_m)\}))\\
\leq dim(span(\{a_1, a_2, ...,a_m\})) + span(\{b_1, b_2, ...,b_m\}))\\ 
\leq dim(span(\{a_1, a_2, ..., a_m\})) + dim(span(\{b_1, b_2, ..., b_m\})) \tag{5} 
\end{multline}
\end{center}

\[
\implies rank(A+B) \leq rank(A) + rank(B) \tag{6}
\]\\

In $eq(5)$ we have used the subadditivity property of dimension of sub-spaces according to which for two sub-spaces $W_1$ and $W_2$ of a vector space $V$, we have \[dim(W_1+W_2) \leq dim(W_1)+dim(W_2) \tag{7}\]

Now, we shall prove this above property.
Consider $W_1$ and $W_2$ to be sub-spaces of the vector space $V$. Let $\beta_1$ and $\beta_2$ be the bases of $W_1$ and $W_2$ respectively.\\

Then,
\[\beta_1 \cap \beta_2 = basis(W_1 \cap W_2) \tag{8}\]
and,
\[\beta_1 \cup \beta_2 = basis(W_1 + W_2) \tag{9}\]

Thus, from $eq(9)$ we have
\[dim(W_1+W_2) = |\beta_1 \cup \beta_2|\]
\[\implies dim(W_1+W_2) = |\beta_1| + |\beta_2| - |\beta_1 \cap \beta_2|\]
\[\implies dim(W_1+W_2) = dim(W_1) + dim(W_2) - dim(W_1 \cap W_2) \tag{10}\]\\

Since, $ dim(W_1 \cap W_2) \geq 0$ we can conclude from $eq(10)$ that
\[dim(W_1+W_2) \leq dim(W_1) + dim(W_2) \tag{11}\]
which finally proves the subaddivity property used in $eq(5)$ and the claim made in $eq(7)$.\\ \\

\textbf{Hence}, having proved $eq(7)$ we can finally conclude from $eq(6)$ that we have \textbf{proven} that for two $m \times n$ matrices $A$ and $B$,  $$rank(A + B) \leq rank(A) + rank(B)$$\\

\begin{center}
    \textbf{Q.E.D.}
\end{center}

\newpage

\section*{Problem 2}
Show that the vectors $B_1=\{(1,1,1), (1,2,3),(1,4,9)\}$ are linearly independent in ${\mathbb C}^3$. Write the vector $v = (1+i,−2-i,5)$ as a linear combination of the three basis vectors. 


\subsection*{Solution}

Consider the given vectors $\vec{b_1} = (1, 1, 1)^t, \, \vec{b_2} = (1, 2, 3)^t \, and \, \vec{b_3} = (1, 4, 9)^t$ where $\vec{b_1}, \vec{b_2} \, and \, \vec{b_3} \in \mathbb{C}^3 $\\

Proving that the given vectors are linearly independent: \\

If the given vectors are linearly independent, then $\exists \, c_1, c_2, c_2$ $\in \mathbb{C}$ such that
\begin{equation}
	\Sigma c_i \vec{b_i} = \phi \Longrightarrow c_i = 0 \, \,  \forall  i \in \{ 1, 2, 3\}
\end{equation}
\begin{equation}
where \, \, \phi = (0, 0, 0)^t 
\end{equation}
\begin{equation}
	c_1 \vec{b_1} + c_2 \vec{b_2} + c_3 \vec{b_3} = \phi\\
\end{equation}
\begin{equation}
	c_1 
	\begin{pmatrix}
	1 \\ 1 \\ 1 
	\end{pmatrix}
	+ c_2 
	\begin{pmatrix}
	1 \\ 2 \\ 3
	\end{pmatrix}		
	+ c_3 
	\begin{pmatrix}
	1 \\ 4 \\ 9
\end{pmatrix}		
	= \phi
\end{equation}
\begin{equation}
	\begin{pmatrix}
	c_1 \\ c_1 \\ c_1
	\end{pmatrix}
	+
	\begin{pmatrix}
	c_2 \\ 2c_2 \\ 3c_2
	\end{pmatrix}
	+
	\begin{pmatrix}
	c_3 \\ 4 c_3 \\ 9 c_3
	\end{pmatrix}
	= \phi
\end{equation}
\begin{equation}
\begin{pmatrix}
	c_1 + c_2 + c_3 \\ c_1 + 2 c_2 + 4 c_3 \\ c_1 + 3 c_2 + 9 c_3
\end{pmatrix}
= \phi
\end{equation}

Solving equations in $c_1, c_2$ and $c_3$:
\begin{equation} \label{eq:q2_1}
	c_1 + c_2 + c_3 = 0
\end{equation}
\begin{equation} \label{eq:q2_2}
	c_1 + 2 c_2 + 4 c_3 = 0
\end{equation}
\begin{equation} \label{eq:q2_3}
	c_1 + 3 c_2 + 9 c_3 = 0
\end{equation}
Rearranging $eq(\ref{eq:q2_1})$ we get, \\
\begin{equation} \label{eq:q2_4}
	c_1 = -(c_2 + c_3)
\end{equation}\\ 
Subtracting $eq(\ref{eq:q2_1})$ from $eq(\ref{eq:q2_2})$, we get
\begin{equation}
	c_2 + 3c_3 = 0
\end{equation}
\begin{equation} \label{eq:q2_5}
	c_2 = -3c_3 
\end{equation}
Substituting $c_2$ from $eq(\ref{eq:q2_5})$ in $eq(\ref{eq:q2_1})$, we get \\
\begin{equation} \label{eq:q2_6}
	c_1 = 2c_3
\end{equation}
Substituting the values of $c_1$ from $eq(\ref{eq:q2_6})$ and $c_2$ from $eq(\ref{eq:q2_5})$ in $eq(\ref{eq:q2_3})$
\begin{equation} \label{eq:q2_7}
	2c_3 + 3(-3c_3) + 9c_3 = 0 \, \Longrightarrow c_3 = 0
\end{equation} 

Hence from $eq(\ref{eq:q2_5}), \, eq(\ref{eq:q2_6}) \, and \, eq(\ref{eq:q2_7})$ we get:	
\[c_1, c_2, c_3 = 0\]
\textbf{Hence, we get all the 3 vectors are linearly independent as $ c_1 \vec{b_1} + c_2 \vec{b_2} + c_3 \vec{b_3} = \phi$ implies $c_1, \, c_2, \, c_3 = 0  $.}\\
\\
Expressing the given vector $\vec{v}$ as a linear combination of $\vec{b_1}, \, \vec{b_2}, \, \vec{b_3}$: \\

\begin{equation}
	Let\ \  \vec{v} = d_1 \vec{b_1} + d_2 \vec{b_2} + d_3 \vec{b_3}
\end{equation}
\begin{equation}
where \, \, d_1, d_2, d_3 \in \mathbb{C}
\end{equation}
\begin{equation}
	d_1 
	\begin{pmatrix}
	1 \\ 1 \\ 1 
	\end{pmatrix}
	+ d_2 
	\begin{pmatrix}
	1 \\ 2 \\ 3
	\end{pmatrix}		
	+ d_3 
	\begin{pmatrix}
	1 \\ 4 \\ 9
\end{pmatrix}		
	= 
	\begin{pmatrix}
		1+ \iota \\ 2 - \iota \\ 5
	\end{pmatrix}
\end{equation}
\begin{equation}
	\begin{pmatrix}
	d_1 \\ d_1 \\ d_1
	\end{pmatrix}
	+
	\begin{pmatrix}
	d_2 \\ 2d_2 \\ 3d_2
	\end{pmatrix}
	+
	\begin{pmatrix}
	d_3 \\ 4 d_3 \\ 9 d_3
	\end{pmatrix}
	= 
	\begin{pmatrix}
		1+ \iota \\ 2 - \iota \\ 5
	\end{pmatrix}
\end{equation}
\begin{equation}
\begin{pmatrix}
	d_1 + d_2 + d_3 \\ d_1 + 2 d_2 + 4 d_3 \\ d_1 + 3 d_2 + 9 d_3
\end{pmatrix}		
	= 
	\begin{pmatrix}
		1+ \iota \\ 2 - \iota \\ 5
	\end{pmatrix}
\end{equation}
Solving equations in $d_1, d_2$ and $d_3$:
\begin{equation} \label{eq:q2_8}
	d_1 + d_2 + d_3 = 1 + \iota
\end{equation}
\begin{equation} \label{eq:q2_9}
	d_1 + 2 d_2 + 4 d_3 = 2 - \iota
\end{equation}
\begin{equation} \label{eq:q2_10}
	d_1 + 3 d_2 + 9 d_3 = 5
\end{equation}

Subtracting $eq(\ref{eq:q2_8})$ from $eq(\ref{eq:q2_9})$, we get
\begin{equation} \label{eq:q2_11}
	d_2 + 3 d_3 = 1 - 2\iota
\end{equation}

Subtracting $eq(\ref{eq:q2_9})$ from $eq(\ref{eq:q2_10})$, we get
\begin{equation} \label{eq:q2_12}
	d_2 + 5 d_3 = 3 + \iota
\end{equation} 
Subtracting $eq(\ref{eq:q2_11})$ from $eq(\ref{eq:q2_12})$, we get
\begin{equation}
	2d_3 = 2 + 3 \iota
\end{equation}
\begin{equation}
	d_3 = \frac{2 + 3 \iota}{2}
\end{equation}
Substituting the value of $d_3$ back in $eq(\ref{eq:q2_11})$, we get
\begin{equation}
	d_2 = -\frac{13\iota + 4}{2}
\end{equation}
Substituting value of $d_3, \, d_2$ back in $eq(\ref{eq:q2_10})$, we get
\begin{equation}
	d_1 = 6\iota + 2
\end{equation}
\textbf{The required linear combination is:} 
\begin{equation}
	\begin{pmatrix}
	1 + \iota \\ 2 - \iota \\ 5
	\end{pmatrix}
	=
	(6\iota + 2)
	\begin{pmatrix}
	1 \\ 1 \\ 1
	\end{pmatrix}
	-\frac{13\iota + 4}{2}
	\begin{pmatrix}
	1 \\ 2 \\ 3
	\end{pmatrix}
	+
	\frac{2 + 3\iota}{2}
	\begin{pmatrix}
	1 \\ 4 \\ 9
	\end{pmatrix}
\end{equation}
 
\end{document}
